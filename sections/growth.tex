\section{Competition Model of the Growth of Fungus Community}\label{sec:LV}


\subsection{Lotka-Volterra Model for Dual-Species System}

In the early twentieth century, Alfred Lotka and Vito Volterra simultaneously derived a model that described how competition affects population growth from the perspective of differential equations, which is now known as the Lotka-Volterra, or LV model.

First, for single population, if the resources are sufficient enough, the population size, or similarly, density, is expected to grow in the form of

\begin{equation}\label{eq:exp}
    \frac{\mathrm{d}x_1}{\mathrm{d}t} = r_1x_1.
\end{equation}

In which, the intrinsic rate of increase $r$ is the per capita growth rate that could potentially be realized by a fungus population. However, unlimited growth is impossible in reality, environment itself has retardant effect on the population. Take the bounded carrying capacity into consideration, obtains

\begin{equation}\label{eq:logistic}
    \frac{\mathrm{d}x_1}{\mathrm{d}t} =
    r_1x_1\left(1 - \frac{x_1}{K_1}\right).
\end{equation}

The extra term places a limit on exponential growth, and the resulting equation represents the logistic growth.

In a dual-species system, the resistance against the growth of one population comes not only from the environment, but also another specie. It is natural that, when specie 2 has larger population density, specie 1 will be in face of larger competitive pressure, and similar for species 2. Such notion is also capable for intra-specie competition. Combining the inter-species and intr-specie competition into \eqref{eq:logistic}, obtains the differential equations set

\begin{equation}\label{eq:dual}
    \left\{\begin{aligned}     &
        \frac{\mathrm{d}x_1}{\mathrm{d}t} =
        r_1x_1\left(1 - \frac{\alpha_{11}x_1 + \alpha_{12}x_2}{K_1}\right) \\ &
        \frac{\mathrm{d}x_2}{\mathrm{d}t} =
        r_2x_2\left(1 - \frac{\alpha_{22}x_2 + \alpha_{21}x_1}{K_2}\right).
    \end{aligned}\right.
\end{equation}

Which is the model for dual-species competition. Note that, neither Lotka nor Volterra explained this equation from the perspective of ecology, it is Tilman who made a convincing interpretation. His resource-based model included resources requirement, consumption and supply, which is concise and rigorous but not the topic of this literature.


\subsection{Extended Lotka-Volterra Model for Multi-Species}

Based on LV model, we extended \eqref{eq:dual} to a multi-species system with $n$ various populations.

\begin{equation}\label{eq:dual}
    \left\{\begin{aligned}     &
    \frac{\mathrm{d}x_1}{\mathrm{d}t} =
    r_1x_1\left(1 - \frac{\sum_{j=1}^n \alpha_{1j}x_j}{K_1}\right) \\ & \cdots \\ &
    \frac{\mathrm{d}x_i}{\mathrm{d}t} =
    r_ix_i\left(1 - \frac{\sum_{j=1}^n \alpha_{ij}x_j}{K_i}\right) \\ & \cdots \\ &
    \frac{\mathrm{d}x_n}{\mathrm{d}t} =
    r_nx_n\left(1 - \frac{\sum_{j=1}^n \alpha_{nj}x_j}{K_n}\right)
    \end{aligned}\right.
\end{equation}

The equations set can be simplified with matrix differential equations as

\begin{equation}\label{eq:mat}
    \frac{\mathrm{d}\boldsymbol{x}}{\mathrm{d}t} =
    \boldsymbol{r}\cdot\boldsymbol{x}
    \left[\boldsymbol{1} - (\boldsymbol{Ax})\cdot\hat{\boldsymbol{K}}\right].
\end{equation}

In which the matrix $\hat{\boldsymbol{K}}$ is the element-wise reciprocal of $\boldsymbol{K}$, that is, $\hat{\boldsymbol{K}}_{ij}\boldsymbol{K}_{ij} = 1$. 

The growth process of fungus community considering competition is characterized by this first-order linear ordinary differential equations group, with $n$ unknown functions. Note that while the analytical can be difficult to be found, we can still figure out the development of the system using computer numerically.

Consider a quad-species system, with the parameters shown in table \ref{tb:quad-para}.

\begin{table}
    \begin{center}
        \caption{The parameters}
        \begin{tabular}{c|cccccc}
            \toprule
                  & $r_i$ & $K_i$ & $\alpha_{i1}$ & $\alpha_{i2}$ & $\alpha_{i3}$ & $\alpha_{i4}$ \\
            \midrule
            $x_1$ &       &       &               &               &               &               \\
            $x_2$ &       &       &               &               &               &               \\
            $x_3$ &       &       &               &               &               &               \\
            $x_4$ &       &       &               &               &               &               \\
            \bottomrule
        \end{tabular}\label{tb:quad-para}
    \end{center}
\end{table}

% TODO: Really that hard? Is there a solution?


% \subsection{Numerical}
