\begin{center}
    \Large\bf Competition, Adaptation and Biodiversity:\\How Fungi Support Our Biosphere
\end{center}

Fungi are prominent components of terrestrial ecosystems in terms of biomass and diversity, and they influence almost every aspect of terrestrial ecosystem functioning. They are the dominant decomposers of organic plant material, with direct consequences for global carbon and nutrient dynamics \cite{Maynard-data}.

In this literature, based on the experimental results of fungi traits provided by Daniel S. Maynard et al., and that of fungi decomposition rate provided by Nicky Lustenhouwer et al., we discussed the decay ability of fungi community, and how environmental conditions affect it. The problem is separated into 4 parts.

Firstly, we construct the model describing the relation between the decomposition rate and hyphal extension rate, moisture tolerance for fungi isolate with straightforward linear regression fitting. The decay ability of a fungi community with respect to the community-weighted extension rate is also specified.

Secondly, we extend the traditional \textbf{Lotka-Volterra model} to multispecific case, and further adapted the equation for the fungi community incorporating hyphal extension rate and moisture tolerance.

Thirdly, we introduced \textbf{Markov chain} model to interspecific competitive biological configuration, and inferred the homeostasis of the fungi community. The attempt of applying a time discrete, state status discrete model to a continuous circumstance is bold and novel.

Finally, we build the model describing how environmental conditions affect the fungi community, and specified the relation between the range of variation of moisture in the local environment and fungi hyphal extension rate.

The models are combined together to form an intact solution for predicting the decay ability of a fungi community. Based on this, we discovered that certain combinations of fungi species can persist in corresponding climatic conditions stably, and may boost the decay ability compared with community merely consists of more combative fungi and with less biodiversity. As the environmental conditions changes, the fungi community may be affected intensively in short term, but in the perspective of long term, the fungi community will reach new homeostasis depends on the environmental conditions. In addition to this, larger biodiversity also means the community is more robust against  the variability in the local environment.

\begin{center}
    \textbf{Keywords:} fungi, interspecific competition, biological degredation, biodiversity
\end{center}