\section{Combined Model of the Decay Ability Regarding Environment}\label{sec:com}

In this section, the decay ability model, the growth model describing the composition of the fungi community, and the environmental factors are combined to determine the behaviors of the fungi community in breakdown process.

\subsection{Combining the models}

In \eqref{eq:dcomm}, replace the hyphal extension rate with \eqref{eq:env}, obtains

\begin{equation}\label{eq:dcomm-env}
\hat{D}_\text{comm} =
    k_3\frac{\sum_{i=1}^n \hat{r}_i}x_i{\sum_{i=1}^n x_i} =
    k_3\dfrac{\sum\limits_{i=1}^n \dfrac{d_ir_ix_i}
    {w + d_i}}{\sum\limits_{i=1}^n x_i}.
\end{equation}

In \eqref{eq:LV-adapt}, conduct the same substitution, obtains

\begin{equation}\label{eq:LV-env}
    \frac{\mathrm{d}x_i}{\mathrm{d}t} =
    \frac{d_ir_i}{w + d_i}x_i\exp\left(
            1 - \frac{\prod_{j=1}^n m_j}{m_i^n}\cdot
            \frac{\prod_{j=1}^n x_j}{x_i^n}
        \right).
\end{equation}

Also, the transition matrix redefined in \eqref{eq:gamma} is again transformed into

\begin{equation}\label{eq:markov-env}
    \hat{\gamma}_{ij} =
    \frac{e^{- \hat{r}_i / \hat{r}_j}}{\sum_{k=1}^n e^{- \hat{r}_i / \hat{r}_k}} =
    \frac{\exp\left(
        -\left.\dfrac{d_ir_i}{w + d_i} \middle/ \dfrac{d_jr_j}{w + d_j}\right.
        \right)}
    {\sum\limits_{k=1}^n\exp\left(
        -\left.\dfrac{d_ir_i}{w + d_i} \middle/ \dfrac{d_kr_k}{w + d_k}\right.
        \right)}.
\end{equation}

Equation \eqref{eq:dcomm-env} determines the ability of a fungi community to decompose ground litter and woody fibers. In which, $r_i$, $d_i$ are the traits of the fungus, $w$ is determined by the environmental condition, $k_3$ is a coefficient and is determined in data fitting as shown in \ref{tb:para-year}. $x_i$ is the only unknown variable representing the biomass of different species of fungus, and can be inferred numerically through \eqref{eq:LV-env}, in which $m_i$ is the trait owned by fungus.



\subsection{Solution of the model at different climatic conditions}

Since a explicit analytical solution cannot be attained from \eqref{eq:LV-env}, yet from model defined in section \ref{sec:markov}, we choose all 34 fungi species with available traits data from [], and evaluate their behaviors under different climatic conditions. For the purpose of brevity, Markov chain model is chosen.

According to the test, we find that, some combinations of fungi species are able to persist stably. Such combinations varies for different climates. In general, fungi community with larger biodiversity behaves more efficient in decomposing ground litter and woody fibers.

Moreover, when environmental conditions change, in short term the community might be influenced intensively. However, in the long term, the community will again reach a new homeostasis, which is more adaptive to the new environment.
