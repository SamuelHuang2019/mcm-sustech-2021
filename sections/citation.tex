\section{Citation examples}

BibTeX provides for the storage of all references in an external, flat-file database. (BibLaTeX uses this same syntax.) This database can be referenced in any LaTeX document, and citations made to any record that is contained within the file. This is often more convenient than embedding them at the end of every document written. The BibTeX file for this document is \href{biblo.bib}{biblo.bib}. BibTeX file can be ceated by \href{https://www.zotero.org/}{zotero}.

To actually cite a given document is very easy. Go to the point where you want the citation to appear, and use the following: 
\textcolor{blue}{\textbackslash cite\{keyword\}}, where the \textcolor{blue}{keyword} is that of the bibitem you wish to cite. When LaTeX processes the document, the citation will be cross-referenced with the bibitems and replaced with the appropriate number citation. 

Citation examples: article \cite{Beauregard2005}, book \cite{Hicks2006}, webpage \cite{spotcrime,doboszczak}.
