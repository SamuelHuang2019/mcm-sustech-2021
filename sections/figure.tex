\section{Figure example}\label{sec:figure}

A simple figure example is shown as figure \ref{fig:mathmodels}.

\begin{figure}[!htb]
\centering
\includegraphics[width=0.6\textwidth]{MATHmodels.pdf}
\caption{Figure example}\label{fig:mathmodels}
\end{figure}

For two independent side-by-side figures, you can use two minipages inside a figure enviroment. Here's an example, shown as figure \ref{fig:h_example_19} and \ref{fig:h_example_82}.
\begin{figure}[!htb]
\begin{minipage}[t]{0.5\linewidth}
\centering
\includegraphics[width=0.8\textwidth]{h_example_19.pdf}
\caption{Figure example}\label{fig:h_example_19}
\end{minipage}
\begin{minipage}[t]{0.5\linewidth}
\centering
\includegraphics[width=0.8\textwidth]{h_example_82.pdf}
\caption{Figure example}\label{fig:h_example_82}
\end{minipage}
\end{figure}


TikZ and PGF are TeX packages for creating graphics programmatically. TikZ is build on top of PGF and allows you to create sophisticated graphics in a rather intuitive and easy manner. More about tikz, see 
\begin{itemize}
\item TikZ and PGF:  \url{http://www.texample.net/tikz}
\item PGFPlots on Sourceforge:  \url{http://pgfplots.sourceforge.net}
\item PGFPlots examples:  \url{http://pgfplots.net/tikz/examples/}
\end{itemize}
Here are a tikzpicture example (figure \ref{fig:tikz}) and a pgfplots example (figure \ref{fig:pgfplots}).
\begin{figure}[!htb]
\begin{minipage}[t]{0.5\linewidth}
\centering
\input{figures/graph-tikz.tikz}
\caption{tikzpicture example. }\label{fig:tikz}
\end{minipage}
\begin{minipage}[t]{0.5\linewidth}
\centering
\centering
\input{figures/plot-pgf.tikz}
\caption{pgfplots example}\label{fig:pgfplots}
\end{minipage}
\end{figure}
